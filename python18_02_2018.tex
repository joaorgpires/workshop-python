%%
% Please see https://bitbucket.org/rivanvx/beamer/wiki/Home for obtaining beamer.
%%
\documentclass{beamer}

\mode<presentation> {
%\usetheme{default}
%\usetheme{AnnArbor}
%\usetheme{Antibes}
%\usetheme{Bergen}
%\usetheme{Berkeley}
%\usetheme{Berlin}
%\usetheme{Boadilla}
%\usetheme{CambridgeUS}
%\usetheme{Copenhagen}
%\usetheme{Darmstadt}
%\usetheme{Dresden}
%\usetheme{Frankfurt}
%\usetheme{Goettingen}
%\usetheme{Hannover}
%\usetheme{Ilmenau}
%\usetheme{JuanLesPins}
%\usetheme{Luebeck}
%\usetheme{Madrid}
%\usetheme{Malmoe}
%\usetheme{Marburg}
%\usetheme{Montpellier}
%\usetheme{PaloAlto}
%\usetheme{Pittsburgh}
%\usetheme{Rochester}
%\usetheme{Singapore}
%\usetheme{Szeged}
\usetheme{Warsaw}
\usecolortheme{orchid}
\setbeamertemplate{navigation symbols}{}
\setbeamertemplate{section in toc}[ball unnumbered]
\setbeamertemplate{subsection in toc}[ball unnumbered]
}

\usepackage[utf8]{inputenc}
\usepackage[portuguese]{babel}
%\usepackage{listings}
%\usepackage{pythonhighlight}
\usepackage{minted}
\usepackage{fontawesome}

\author{João Rebelo Pires}
\institute[DCC]{NUCC-FCUP (DCC)}
\title[Python com Class]{Python: Um Workshop Orientado a Objectos}
\date{18 de Fevereiro de 2018}

\newminted{python}{fontsize=\normalsize,
linenos=false,
numbersep=10pt,
gobble=0,
frame=none,
breaklines}

\newminted{pycon}{bgcolor=bg, linenos=true, tabsize=4}
\setminted[pycon]{autogobble}

\begin{document}
	%-------------------------------------------------
	%--- Cover
	%-------------------------------------------------
	\begin{frame}
		\titlepage
	\end{frame}
	
	%-------------------------------------------------
	%--- Table of Contents
	%-------------------------------------------------
	\begin{frame}
		\frametitle{Conteúdos}
		\tableofcontents
	\end{frame}
	
	%-------------------------------------------------
	%--- Section 1: Language overview
	%-------------------------------------------------
	\section{História}
	%-------------------------------------------------
	\begin{frame}
		\frametitle{História}
		\begin{itemize}
			\item Criado por Guido van Rossum; \pause
			\item Lançado pela primeira vez em 1991; \pause
			\item Nome é referência aos Monty Python; \pause
			\item Linguagem multi-paradigma; \pause
			\item Fornece construtores que permitem programação "limpa" a qualquer escala.
		\end{itemize}
	\end{frame}
	
	%-------------------------------------------------
	%--- Section 2: Simple programs
	%-------------------------------------------------
	\section{Programas Simples}
	%-------------------------------------------------
	\subsection*{Hello World (típico...)}
	\begin{frame}[fragile]
		\frametitle{Hello World (típico...)}
		Tão simples como:
		\vspace{10pt}
		\inputminted{python}{helloWorld.py}
	\end{frame}
	
	\subsection*{Somar dois números introduzidos pelo utilizador}
	\begin{frame}[fragile]
		\frametitle{Somar dois números introduzidos pelo utilizador}
		\inputminted{python}{sumTwoInts.py}
	\end{frame}
	
	\subsection*{Usando listas}
	\begin{frame}[fragile]
		\frametitle{Usando listas}
		\inputminted{python}{lists.py}
	\end{frame}
	
	\subsection*{Prints com formatação}
	\begin{frame}[fragile]
		\frametitle{Prints com formatação}
		\inputminted[breaklines]{python}{formattedPrinting.py}
	\end{frame}
	
	\subsection*{Questões}
	\begin{frame}
		\begin{center}\huge{Questões?}\end{center}
	\end{frame}
	
	%-------------------------------------------------
	%--- Section 3: Class (OOP)
	%-------------------------------------------------
	\section{Classes}
	%-------------------------------------------------
	\subsection*{Introdução}
	\begin{frame}[fragile]
		\frametitle{Introdução}
		\begin{itemize}
			\item Objectos; \pause
			\item Instâncias; \pause
			\item Métodos; \pause
			\item Atributos; \pause
			\item Herança.
		\end{itemize}
	\end{frame}
	
	\subsection*{Métodos especiais para objectos}
	\begin{frame}[fragile]
		\frametitle{Métodos especiais para objectos}
		\begin{itemize}
			\item \mint{python}|__init__()| \pause
			\item \mint{python}|__del__()| \pause
			\item \mint{python}|__repr__()| \pause
			\item \mint{python}|__str__()| \pause
			\item \mint{python}|__eq__()| \pause
			\item \mint{python}|__ne__()| \pause
			\item \mint{python}|__hash__()|
		\end{itemize}
	\end{frame}
	
	\subsection*{Exemplo}
	\begin{frame}[fragile]
		\frametitle{Exemplo - class \texttt{Trick}}
		\inputminted[breaklines, firstline=4, lastline=13]{python}{classExample.py}
	\end{frame}
	
	\begin{frame}[fragile]
		\frametitle{Exemplo - class \texttt{Trick}}
		\inputminted[breaklines, firstline=15, lastline=24]{python}{classExample.py}
	\end{frame}
	
	\begin{frame}[fragile]
		\frametitle{Exemplo - class \texttt{Dog}}
		\inputminted[breaklines, firstline=27, lastline=35]{python}{classExample.py}
	\end{frame}
	
	\begin{frame}[fragile]
		\frametitle{Exemplo - class \texttt{Dog}}
		\inputminted[breaklines, firstline=37, lastline=45]{python}{classExample.py}
	\end{frame}
	
	\begin{frame}[fragile]
		\frametitle{Exemplo - class \texttt{Dog}}
		\inputminted[breaklines, firstline=47, lastline=57]{python}{classExample.py}
	\end{frame}
	
	\begin{frame}[fragile]
		\frametitle{Exemplo - class \texttt{Dog}}
		\inputminted[breaklines, firstline=59, lastline=66]{python}{classExample.py}
	\end{frame}
	
	\begin{frame}[fragile]
		\frametitle{Exemplo - class \texttt{Competition}}
		\inputminted[breaklines, firstline=69, lastline=76]{python}{classExample.py}
	\end{frame}
	
	\begin{frame}[fragile]
		\frametitle{Exemplo - class \texttt{Competition}}
		\inputminted[breaklines, firstline=78, lastline=87]{python}{classExample.py}
	\end{frame}
	
	\begin{frame}[fragile]
		\frametitle{Exemplo - class \texttt{Competition}}
		\inputminted[breaklines, firstline=88, lastline=94]{python}{classExample.py}
	\end{frame}
	
	\begin{frame}[fragile]
		\frametitle{Exemplo - class \texttt{Competition}}
		\inputminted[breaklines, firstline=96, lastline=103]{python}{classExample.py}
	\end{frame}
	
	\begin{frame}[fragile]
		\frametitle{Exemplo - class \texttt{Competition}}
		\inputminted[breaklines, firstline=104, lastline=107]{python}{classExample.py}
	\end{frame}
	
	\begin{frame}[fragile]
		\frametitle{Exemplo - class \texttt{Competition}}
		\inputminted[breaklines, firstline=109, lastline=113]{python}{classExample.py}
	\end{frame}
	
	\begin{frame}[fragile]
		\frametitle{Exemplo - class \texttt{Competition}}
		\inputminted[breaklines, firstline=115, lastline=124]{python}{classExample.py}
	\end{frame}
	
	\begin{frame}[fragile]
		\frametitle{Exemplo - class \texttt{Competition}}
		\inputminted[breaklines, firstline=125, lastline=133]{python}{classExample.py}
	\end{frame}
	
	\begin{frame}[fragile]
		\frametitle{Exemplo - Utilização}
		\inputminted[breaklines, firstline=136, lastline=136]{python}{classExample.py} \pause
		\inputminted[breaklines, firstline=139, lastline=139]{python}{classExample.py} \pause
		\inputminted[breaklines, firstline=140, lastline=140]{python}{classExample.py} \pause
		\inputminted[breaklines, firstline=144, lastline=144]{python}{classExample.py} \pause
		\inputminted[breaklines, firstline=145, lastline=145]{python}{classExample.py} \pause
		\inputminted[breaklines, firstline=150, lastline=150]{python}{classExample.py} \pause
		\inputminted[breaklines, firstline=151, lastline=151]{python}{classExample.py}
	\end{frame}
	
	\begin{frame}[fragile]
		\frametitle{Exemplo - Utilização}
		\mint{pycon}{
			>>> comp.podium()
		} \pause
		First Place:  rufus with 1 point(s)
	\end{frame}
	
	\section*{}
	\begin{frame}
		\begin{center}
			Vai um live coding?
			\begin{figure}
				\includegraphics[scale=0.17]{images/sir}
			\end{figure}
			{\faGithub} jrbartowski \hspace{1cm} {\faEnvelope} joaorgpires@gmail.com
		\end{center}
	\end{frame}

\end{document}
